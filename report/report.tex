% report.tex

\documentclass[a4paper,11pt]{article}
\usepackage{hyperref}
\usepackage{enumitem}
\usepackage{varwidth}
\usepackage{tasks}
% Import packages
\usepackage[a4paper]{geometry}
\usepackage[utf8]{inputenc}
\usepackage{amsmath}
\usepackage{amssymb}
\usepackage{enumerate}
\usepackage{geometry}
 \geometry{
 a4paper,
 total={170mm,257mm},
 left=20mm,
 top=20mm,
 }

\usepackage{graphicx}

\usepackage{listings}

% Change enumerate environments you use letters
\renewcommand{\theenumi}{\alph{enumi}}

% Set title, author name and date
\title{Consensus}
\author{Johannes Jørgensen (jgjo),\\ Kevin Skovgaard Gravesen (kegr),\\ Joakim Andreasen (joaan)} 
\date{\today}

\begin{document} 

\maketitle

\subsection{Introduction}
This small GoLang program is a demo of active replication. A list of ports can be added to a .env file,
which the clients and server (nodes) can use to find their respective ports.

The demo assumes that no messages are lost, which would require some alteration to the code if this was the case.

\subsection{Architecture}
A description of the architecture of the system and its protocols (behaviour), including any protocols used internally between nodes of the system. 
\subsection{Correctness 1.}
Argue whether your implementation satisfies linearisability or sequential consistency. In order to do so, first, you must state precisely what such property is.

If a system is linearizable it means that the systems needs to process operations according to their real-time invocation.
This includes the travel time of messages sent in the system. So if user farther away from a server sends a request,
that request has to be processed before a possible closer invocation.

This demo does not satisfy this requirement because of potential race conditions between the servers (nodes) and clients.
Ofcourse this demo runs locally on the same machine, which minimizes the risk for out of order handling of the requests.

But if the system were deployed globally, theses race conditions would become every more problematic.


A sequential consistant system does not need the real-time order, but the timeline of operations still needs to be the same on every client.

This is easier to deal with, and if the nodes in the demo locked a central database, then the system could be sqeuential consistant.
Altough that would mean that there is a single point of failure on the database that also would need to be accounted for.

\subsection{Correctness 2.}
An argument that your protocol is correct in the absence and the presence of failure

\newpage
\subsection*{Link to Github repository}

\href{https://github.com/ITU-DISYS2024-CENTRALIZEDSYSTEMS/Consensus}{https://github.com/ITU-DISYS2024-CENTRALIZEDSYSTEMS/Consensus}

\subsection*{Appendix}

\subsubsection*{Logs - Peer #1}
\begin{lstlisting}[basicstyle=\ttfamily\footnotesize]

\end{lstlisting}

\subsubsection*{Logs - Peer #2}
\begin{lstlisting}[basicstyle=\ttfamily\footnotesize]

\end{lstlisting}

\subsubsection*{Logs - Peer #3}
\begin{lstlisting}[basicstyle=\ttfamily\footnotesize]

\end{lstlisting}
\end{document}